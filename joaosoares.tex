%%%%%%%%%%%%%%%%%%%%%%%%%%%%%%%%%%%%%%%
% Deedy - One Page Two Column Resume
% LaTeX Template
% Version 1.2 (16/9/2014)
%
% Original author:
% Debarghya Das (http://debarghyadas.com)
%
% Original repository:
% https://github.com/deedydas/Deedy-Resume
%
% IMPORTANT: THIS TEMPLATE NEEDS TO BE COMPILED WITH XeLaTeX
%
% This template uses several fonts not included with Windows/Linux by
% default. If you get compilation errors saying a font is missing, find the line
% on which the font is used and either change it to a font included with your
% operating system or comment the line out to use the default font.
% 
%%%%%%%%%%%%%%%%%%%%%%%%%%%%%%%%%%%%%%
% 
% TODO:
% 1. Integrate biber/bibtex for article citation under publications.
% 2. Figure out a smoother way for the document to flow onto the next page.
% 3. Add styling information for a "Projects/Hacks" section.
% 4. Add location/address information
% 5. Merge OpenFont and MacFonts as a single sty with options.
% 
%%%%%%%%%%%%%%%%%%%%%%%%%%%%%%%%%%%%%%
%
% CHANGELOG:
% v1.1:
% 1. Fixed several compilation bugs with \renewcommand
% 2. Got Open-source fonts (Windows/Linux support)
% 3. Added Last Updated
% 4. Move Title styling into .sty
% 5. Commented .sty file.
%
%%%%%%%%%%%%%%%%%%%%%%%%%%%%%%%%%%%%%%%
%
% Known Issues:
% 1. Overflows onto second page if any column's contents are more than the
% vertical limit
% 2. Hacky space on the first bullet point on the second column.
%
%%%%%%%%%%%%%%%%%%%%%%%%%%%%%%%%%%%%%%

\documentclass[]{joaosoares-resume}
\usepackage{ragged2e}

\begin{document}

%%%%%%%%%%%%%%%%%%%%%%%%%%%%%%%%%%%%%%
%     LAST UPDATED DATE
%%%%%%%%%%%%%%%%%%%%%%%%%%%%%%%%%%%%%%
\lastupdated

%%%%%%%%%%%%%%%%%%%%%%%%%%%%%%%%%%%%%%
%     TITLE NAME
%%%%%%%%%%%%%%%%%%%%%%%%%%%%%%%%%%%%%%
\namesection{João}{Soares}{
\href{mailto:eng.jmsoares@gmail.com}{eng.jmsoares@gmail.com}, \\
+55 (91) 98220-3284,
Passagem Liberal 245 - Belém do Pará - Brazil, 
25 years old
}

%%%%%%%%%%%%%%%%%%%%%%%%%%%%%%%%%%%%%%
%     COLUMN ONE
%%%%%%%%%%%%%%%%%%%%%%%%%%%%%%%%%%%%%%

\begin{minipage}[t]{0.35\textwidth} 

%%%%%%%%%%%%%%%%%%%%%%%%%%%%%%%%%%%%%%
%     EDUCATION
%%%%%%%%%%%%%%%%%%%%%%%%%%%%%%%%%%%%%%

\section{Education}
%\section{Educação}

\subsection{\href{http://www.portal.ufpa.br/}{Federal University of Para}}
%\subsection{\href{http://www.portal.ufpa.br/}{Universidade Federal do Pará}}
\descript{\href{http://www.fct.ufpa.br/}{B. Eng. in Computer Engineering}}
%\descript{Bacharel em Engenharia da Computação}
\location{Jan 2014 | Belém, PA}
\sectionsep

%%%%%%%%%%%%%%%%%%%%%%%%%%%%%%%%%%%%%%
%     LINKS
%%%%%%%%%%%%%%%%%%%%%%%%%%%%%%%%%%%%%%

\section{Links} 
Skype:// joao-rembrandt@hotmail.com \\
Github:// \href{https://github.com/joaosr}{\bf joaosr} \\
LinkedIn://  \href{https://www.linkedin.com/in/joao-soares}{\bf joao-soares} \\
Twitter://  \href{https://twitter.com/joao_mnl}{\bf @joao\underline{\hspace{.10in}}mnl}\\
\sectionsep

%%%%%%%%%%%%%%%%%%%%%%%%%%%%%%%%%%%%%%
%     CERTIFICATIONS
%%%%%%%%%%%%%%%%%%%%%%%%%%%%%%%%%%%%%%

%\texttt{\section{Certifications}
%\href{https://www.lpi.org/certification/get-certified-lpi/lpic-1-linux-server-professional/}{LPIC1} - 2013 \\
%\href{https://www.goethe.de/en/spr/kup/prf/prf/gb1.html}{Goethe-Zertifikat B1} - 2013
%\sectionsep}

%%%%%%%%%%%%%%%%%%%%%%%%%%%%%%%%%%%%%%
%     SKILLS
%%%%%%%%%%%%%%%%%%%%%%%%%%%%%%%%%%%%%%

\section{Skills}
%\section{Habilidades}

\subsection{Programming}
%\subsection{Programação}
\location{Over 10,000 lines:}
%\location{Acima de 1000 linhas:}
\href{http://www.python.org}{Python} \textbullet{}
\href{http://www.php.net}{JavaScript} \textbullet{}
\href{http://android.com}{Android} \textbullet{}
\href{https://www.oracle.com/java/}{Java}\\
\href{http://www.php.net}{PHP} \textbullet{}
CSS \textbullet{}
\href{http://jinja.pocoo.org/}{Jinja} \textbullet{}
\href{http://www.latex-project.org}{\LaTeX} \\
\location{Over 1,000 lines:}
C \textbullet{}
SQL \textbullet{}
\href{https://mustache.github.io/}{Mustache}\\
\sectionsep

\subsection{Frameworks}
\location{Python:}
\href{https://www.djangoproject.com/}{Django} \textbullet{}
\href{http://flask.pocoo.org/}{Flask} \textbullet{}
\href{https://scrapy.org/}{Scrapy} \\
\location{JavaScript:}
\href{https://angularjs.org}{AngularJS} \textbullet{}
\href{https://backbonejs.org/}{Backbone.js} \textbullet{}
\href{https://jasmine.github.io/}{Jasmine} \\
\href{https://nodejs.org/en/}{NodeJS} \\
\location{PHP:}
\href{https://cakephp.org/}{CakePHP} \\
\location{CSS:}
\href{http://getbootstrap.com/}{Bootstrap} \\
\sectionsep

\subsection{Linux Server}
%\subsection{Servidores Linux}
\location{Distros:}
\href{http://www.debian.org/}{Debian}\textbullet{}
\href{http://www.ubuntu.com/}{Ubuntu}\textbullet{}
\href{http://www.centos.org/}{CentOS} \\
\location{Services:}
%\location{Serviços:}
\href{http://www.postgresql.org/}{PostgreSQL}\textbullet{}
\href{http://www.postgis.net}{PostGIS}\textbullet{}
\href{http://httpd.apache.org/}{Apache}\\
\href{https://www.nginx.com/resources/wiki/}{Nginx}\textbullet{}
\href{https://www.mysql.com/}{MySQL} \textbullet{}
\href{https://realm.io/}{Realm}\\
\href{https://www.mongodb.com/}{MongoDB} \\
\location{Tools:}
%\location{Ferramentas:}
\href{http://www.docker.com/}{Docker} \textbullet{}
\href{http://puppetlabs.com/}{Puppet} \textbullet{}
\href{https://www.ansible.com/}{Ansible} \\
\sectionsep

\subsection{Languages}
\location{Native level:}
Portuguese \\
\location{Professional level:}
English (B2) \\
%\location{Intermediate level:}
%German (B1)
\sectionsep

\section{Conferences}
\location{Attending:}
Python Brazil (PyBr) - \href{http://2016.pythonbrasil.org.br/}{2016}, \href{http://2015.pythonbrasil.org.br/}{2015}, \href{https://2014.pythonbrasil.org.br/}{2014} \\
\href{http://earthenginesummit2015.earthoutreach.org/}{Google Earth Engine User} - 2015 Mountain View, California \\
\location{Teaching:}
Coding Dojo - 2016, 2015
\section{Hobbies}
%\section{Habilidades}
\location{Yoga \textbullet{} \href{https://en.wikipedia.org/wiki/Carimbo}{Carimbó} \textbullet{} \href{http://pynorte.python.org.br}{PyNorte}}

%%%%%%%%%%%%%%%%%%%%%%%%%%%%%%%%%%%%%%
%     COLUMN TWO
%%%%%%%%%%%%%%%%%%%%%%%%%%%%%%%%%%%%%%

\end{minipage}
\hfill
\begin{minipage}[t]{0.64\textwidth}

%%%%%%%%%%%%%%%%%%%%%%%%%%%%%%%%%%%%%%
%     EXPERIENCE
%%%%%%%%%%%%%%%%%%%%%%%%%%%%%%%%%%%%%%
\section{Experience}
\runsubsection{\href{http://terras.agr.br/}{Terras}}
\descript{| Junior Software Developer}
\location{Jul 2015 – Present | Belém, PA}
\location{{\bf Keywords:}
    \href{http://leafletjs.com}{Leaflet},
    \href{https://www.openstreetmap.org}{OpenStreetMap},
    \href{https://www.mapbox.com}{Mapbox},
	\href{http://angularjs.org}{AngularJS},
	\href{http://cakephp.org}{CakePHP},
	\href{http://www.python.org}{Python},
%	\href{https://realm.io/}{Realm},
%	\href{https://developer.android.com/training/volley/index.html}{Volley},
	\href{http://git-scm.com}{Git}.
}

\sectionsep

{\justifying \noindent TERRAS develops services and tools for Brazilian agribusiness, focused on developing applications that draw upon insights from rich spatial data. My responsibilities in its software development team:\par} 

\sectionsep
\begin{tightemize}
\item Create map tools with help of \href{https://www.openstreetmap.org}{OpenStreetMap}, \href{http://leafletjs.com}{Leaflet} and \href{https://www.mapbox.com}{Mapbox} libraries 
\item Set up AngularJS projects and integrate them with build tools (bower.js, gulp.js, karma.js, jasmine.js)
\item Use Scrum and Kanban to plan and manage projects
\item Use development practices such as MVC, Test Drive Development (TDD), Baby Steps and Project Patterns
\item Implement unit test cases and maintain them with the help of test coverage tools
%\item Developer PHP Web Services system using the REST API Patterns
\end{tightemize}
\sectionsep

\runsubsection{\href{http://imazon.org.br/}{Imazon}}
\descript{| Software Development Consultant}
\location{Jan - Jul 2015 | Belém, PA}
\location{{\bf Keywords:}
    \href{http://cloud.google.com/appengine}{GAE},
    \href{http://earthengine.google.com}{GEE},
    \href{http://www.python.org}{Python},
    \href{http://flask.pocoo.org}{Flask},
	\href{http://backbonejs.org}{Backbone.js}.
}

\sectionsep

{\justifying \noindent In this semester I had been working giving support for the Deforestation Analysis Tool web application. In the back-end I added news features with support of the \href{http://flask.pocoo.org}{Flask} web framework, Google App Engine (GAE) and Google Earth Engine (GEE). And in the front-end I had been customizing the web application map with the help of \href{https://backbonejs.org}{Backbone.js} javascript framework.\par}

\sectionsep

\runsubsection{\href{http://imazon.org.br/}{Imazon}}
\descript{| Android Development Consultant}
\location{Jan – Dec 2014 | Belém, PA}
\location{{\bf Keywords:}
    \href{https://eclipse.org}{Eclipse ADT},
    \href{https://developer.android.com/studio/}{Android Studio},
    \href{http://junit.org}{JUnit},
    \href{http://ant.apache.org}{Ant},
    \href{https://subversion.apache.org}{SVN},
    \href{https://jenkins.io}{Jenkins}.
}
\sectionsep
{\justifying \noindent I helped create and maintain the android application \href{https://play.google.com/store/apps/details?id=br.agr.terras.monitoramentopragas}{Monitoramento de Pragas} to monitor activities on agricultural properties. My focus was on:\par}
\sectionsep
\begin{tightemize}
\item Represent GIS informations like polygons of plantation areas, satellite image, pests and disease occurrences in a mobile map application
\item Develop, test and debugg REST API services
%\item Use design patterns and refactoring to increase bug solutions and code readability
\item Travel to some farms to get a close look on the complete user experience
\end{tightemize}
\sectionsep

\runsubsection{\href{http://imazon.org.br/}{Imazon}}
\descript{| Android Mobile Development Trainee}
\location{Jul - Dec 2013 | Belém, PA}
\location{{\bf Keywords:}
    \href{https://developer.android.com/studio/tools/sdk/eclipse-adt.html}{Eclipse ADT},
    \href{http://junit.org}{JUnit},
    \href{http://ant.apache.org}{Ant},
    \href{https://subversion.apache.org}{SVN}.
}
\sectionsep 

{\justifying \noindent I worked with IMAZON’s android app development with IDEs \href{https://developer.android.com/studio/tools/sdk/eclipse-adt.html}{Eclipse ADT} and \href{https://developer.android.com/studio/}{Android Studio}. My focus was on the development of an android application using \href{https://developers.google.com/maps/documentation/android-api}{Android Google Maps API}, \href{http://greenrobot.org/greendao}{GreenDAO}, and \href{https://square.github.io/retrofit}{RETROFIT} framework.\par}

\sectionsep

\runsubsection{\href{http://imazon.org.br/}{Imazon}}
\descript{| Trainee Web Developer}
\location{Jun 2012 - Jul 2013 | Belém, PA}
\location{{\bf Keywords:}
    \href{http://www.djangoproject.com}{Django},
    \href{http://plugins.jquery.com}{JQuery},
    \href{http://www.postgresql.org}{PostgreSQL},
    \href{http://mapserver.org/mapscript/}{Mapscript}.
}
\sectionsep 

{\justifying \noindent I worked as Full-Stack programmer on the Python web framework Django, and was responsible for setting up the web services for the green municipalities project. I was programming the database manager, the domain logic of the application to improve the users’s dataset view like annual deforestation, water access and per capita income of Pará state. I also built a web page for municipality statistics with the help of tools like \href{https://developers.google.com/chart}{Google Charts}, \href{http://plugins.jquery.com}{JQuery} and  \href{https://api.jquery.com/category/ajax}{AJAX}.\par}

\sectionsep

\end{minipage} 

\end{document} % End of resume.
\documentclass{article}

\textit{•}
