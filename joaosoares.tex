%%%%%%%%%%%%%%%%%%%%%%%%%%%%%%%%%%%%%%%
% Deedy - One Page Two Column Resume
% LaTeX Template
% Version 1.2 (16/9/2014)
%
% Original author:
% Debarghya Das (http://debarghyadas.com)
%
% Original repository:
% https://github.com/deedydas/Deedy-Resume
%
% IMPORTANT: THIS TEMPLATE NEEDS TO BE COMPILED WITH XeLaTeX
%
% This template uses several fonts not included with Windows/Linux by
% default. If you get compilation errors saying a font is missing, find the line
% on which the font is used and either change it to a font included with your
% operating system or comment the line out to use the default font.
% 
%%%%%%%%%%%%%%%%%%%%%%%%%%%%%%%%%%%%%%
% 
% TODO:
% 1. Integrate biber/bibtex for article citation under publications.
% 2. Figure out a smoother way for the document to flow onto the next page.
% 3. Add styling information for a "Projects/Hacks" section.
% 4. Add location/address information
% 5. Merge OpenFont and MacFonts as a single sty with options.
% 
%%%%%%%%%%%%%%%%%%%%%%%%%%%%%%%%%%%%%%
%
% CHANGELOG:
% v1.1:
% 1. Fixed several compilation bugs with \renewcommand
% 2. Got Open-source fonts (Windows/Linux support)
% 3. Added Last Updated
% 4. Move Title styling into .sty
% 5. Commented .sty file.
%
%%%%%%%%%%%%%%%%%%%%%%%%%%%%%%%%%%%%%%%
%
% Known Issues:
% 1. Overflows onto second page if any column's contents are more than the
% vertical limit
% 2. Hacky space on the first bullet point on the second column.
%
%%%%%%%%%%%%%%%%%%%%%%%%%%%%%%%%%%%%%%

\documentclass[]{joaosoares-resume}

\begin{document}

%%%%%%%%%%%%%%%%%%%%%%%%%%%%%%%%%%%%%%
%     LAST UPDATED DATE
%%%%%%%%%%%%%%%%%%%%%%%%%%%%%%%%%%%%%%
\lastupdated

%%%%%%%%%%%%%%%%%%%%%%%%%%%%%%%%%%%%%%
%     TITLE NAME
%%%%%%%%%%%%%%%%%%%%%%%%%%%%%%%%%%%%%%
\namesection{João}{Soares}{
\href{mailto:eng.jmsoares@gmail.com}{eng.jmsoares@gmail.com}
}

%%%%%%%%%%%%%%%%%%%%%%%%%%%%%%%%%%%%%%
%     COLUMN ONE
%%%%%%%%%%%%%%%%%%%%%%%%%%%%%%%%%%%%%%

\begin{minipage}[t]{0.35\textwidth} 

%%%%%%%%%%%%%%%%%%%%%%%%%%%%%%%%%%%%%%
%     EDUCATION
%%%%%%%%%%%%%%%%%%%%%%%%%%%%%%%%%%%%%%

\section{Education}
%\section{Educação}

\subsection{\href{http://www.portal.ufpa.br/}{Federal University of Para}}
%\subsection{\href{http://www.portal.ufpa.br/}{Universidade Federal do Pará}}
\descript{\href{http://www.fct.ufpa.br/}{BS in Computer Engineering}}
%\descript{Bacharel em Engenharia da Computação}
\location{Jan 2014 | Belém, PA}
\sectionsep

%%%%%%%%%%%%%%%%%%%%%%%%%%%%%%%%%%%%%%
%     LINKS
%%%%%%%%%%%%%%%%%%%%%%%%%%%%%%%%%%%%%%

\section{Links} 
Skype:// joao-rembrandt@hotmail.com \\
Github:// \href{https://github.com/joaosr}{\bf joaosr} \\
LinkedIn://  \href{www.linkedin.com/in/joao-soares}{\bf joao-soares} \\
Twitter://  \href{https://twitter.com/joao_mnl}{\bf @joao\underline{\hspace{.10in}}mnl}\\
%Stack Exchange://  \href{http://stackexchange.com/users/1687314/willian-paixao}{\bf willian-paixao}
\sectionsep

%%%%%%%%%%%%%%%%%%%%%%%%%%%%%%%%%%%%%%
%     CERTIFICATIONS
%%%%%%%%%%%%%%%%%%%%%%%%%%%%%%%%%%%%%%

%\texttt{\section{Certifications}
%\href{https://www.lpi.org/certification/get-certified-lpi/lpic-1-linux-server-professional/}{LPIC1} - 2013 \\
%\href{https://www.goethe.de/en/spr/kup/prf/prf/gb1.html}{Goethe-Zertifikat B1} - 2013
%\sectionsep}

%%%%%%%%%%%%%%%%%%%%%%%%%%%%%%%%%%%%%%
%     SKILLS
%%%%%%%%%%%%%%%%%%%%%%%%%%%%%%%%%%%%%%

\section{Skills}
%\section{Habilidades}
\subsection{Linux Server}
%\subsection{Servidores Linux}
\location{Distros:}
\href{http://www.debian.org/}{Debian}\textbullet{}
\href{http://www.ubuntu.com/}{Ubuntu}\textbullet{}
\href{http://www.centos.org/}{CentOS} \\
\location{Services:}
%\location{Serviços:}
\href{http://httpd.apache.org/}{Apache}\textbullet{}
\href{https://www.nginx.com/resources/wiki/}{Nginx}\textbullet{}
\href{https://www.mysql.com/}{MySQL} \textbullet{}
\href{http://www.postgresql.org/}{PostgreSQL} \textbullet{}
\href{https://www.mongodb.com/}{MongoDB} \\
\location{Tools:}
%\location{Ferramentas:}
%\href{https://www.openshift.com/}{OpenShift} \textbullet{}
%\href{http://www.linux-kvm.org/}{KVM} \textbullet{}
%\href{http://xenproject.org/}{Xen} \\
\href{http://www.docker.com/}{Docker} \textbullet{}
\href{http://puppetlabs.com/}{Puppet} \textbullet{}
\href{https://www.ansible.com/}{Ansible} \textbullet{}
%\href{http://www.zabbix.com/}{Zabbix} \\
%\href{http://www.nagios.com/}{Nagios} \textbullet{}
%\href{https://www.icinga.org/}{Icinga} \\
\sectionsep

\subsection{Programming}
%\subsection{Programação}
\location{Over 10,000 lines:}
%\location{Acima de 1000 linhas:}
\href{http://www.python.org}{Python} \textbullet{}
\href{http://www.php.net}{JavaScript} \textbullet{}
\href{http://android.com}{Android} \textbullet{}
\href{https://www.oracle.com/java/}{Java} \textbullet{}
\href{http://www.php.net}{PHP} \textbullet{}
CSS \textbullet{}
\href{http://jinja.pocoo.org/}{Jinja} \textbullet{}
\href{http://www.latex-project.org}{\LaTeX} \\
\location{Over 1,000 lines:}
C \textbullet{}
SQL \textbullet{}
\href{https://mustache.github.io/}{Mustache}\\
\sectionsep

\subsection{Frameworks}
\location{Python:}
\href{https://www.djangoproject.com/}{Django} \textbullet{}
\href{http://flask.pocoo.org/}{Flask} \textbullet{}
\href{https://scrapy.org/}{Scrapy} \\
\location{JavaScript:}
\href{https://angularjs.org/}{AngularJS} \textbullet{}
\href{http://backbonejs.org/}{Backbone.js} \textbullet{}
\href{https://jasmine.github.io/}{Jasmine} \textbullet{}
\href{https://nodejs.org/en/}{NodeJS} \\
\location{PHP:}
\href{https://cakephp.org/}{CakePHP} \\
\location{CSS:}
\href{http://getbootstrap.com/}{Bootstrap} \\
\sectionsep

\subsection{Languages}
\location{Native level:}
Portuguese \\
\location{Professional level:}
English (C1) \\
%\location{Intermediate level:}
%German (B1)

%%%%%%%%%%%%%%%%%%%%%%%%%%%%%%%%%%%%%%
%     COLUMN TWO
%%%%%%%%%%%%%%%%%%%%%%%%%%%%%%%%%%%%%%

\end{minipage} 
\hfill
\begin{minipage}[t]{0.64\textwidth}

%%%%%%%%%%%%%%%%%%%%%%%%%%%%%%%%%%%%%%
%     EXPERIENCE
%%%%%%%%%%%%%%%%%%%%%%%%%%%%%%%%%%%%%%
\section{Experience}
\runsubsection{\href{http://terras.agr.br/}{Terras}}
\descript{| Junior Software Development}
\location{Jul 2015 – Present | Belém, PA}
\location{{\bf Keywords:}
    Leaflet,
	AngularJS,
	CakePHP,
	Python, 
	Google Earth Engine
	Git.
}

\sectionsep

TERRAS develops services and tools for Brazilian agribusiness, focused on developing applications that draw upon insights from rich spatial data. My responsibility in its software development team: 

\sectionsep
\begin{tightemize}
\item Set up an AngularJS development environment
\item Integrate projects with build tools (bower.js, gulp.js, karma.js, jasmine.js)
\item Use agile project management tools to plan and manage projects
\item Use good development practices, such as Test Drive Development (TDD), Baby Steps and Project Patterns
\item Implement unit tests cases and maintenance with help of tests coverage tools
\item Create map tools that help customers to create and manage their spatial data
\item Developer PHP Web Services system using the REST API Patterns
\end{tightemize}
\sectionsep

\runsubsection{\href{http://imazon.org.br/}{Imazon}}
\descript{| Software Development Consultant}
\location{Jan - July 2015 | Belém, PA}
\location{{\bf Keywords:}
    \href{https://cloud.google.com/appengine}{GAE},
    \href{https://earthengine.google.com}{GEE},
    \href{http://www.python.org}{Python},
    \href{http://flask.pocoo.org}{Flask},
	\href{http://backbonejs.org}{Backbone.js}
}
\sectionsep
In this semester I had been working giving support for the Deforestation Analysis Tool web application. In the back-end I added news features with support of Flask web framework,  Google App Engine (GAE) and uses Google Earth Engine (GEE). And in front-end I had been customizing the web application map with the help of Backbone.js javascript framework. 
\sectionsep

\runsubsection{\href{http://imazon.org.br/}{Imazon}}
\descript{| Android Development Consultant}
\location{Jan – Dec 2014 | Belém, PA}
\location{{\bf Keywords:}
    \href{https://eclipse.org}{Eclipse ADT},
    \href{https://developer.android.com/studio/}{Android Studio},
    \href{http://junit.org}{JUnit},
    \href{http://ant.apache.org}{Ant},
    \href{https://subversion.apache.org}{SVN},
    \href{https://jenkins.io}{Jenkins}.
}
\sectionsep

I helped created and maintain an \href{https://play.google.com/store/apps/developer?id=TERRAS\%20App\%20Solutions}{android application} to monitor activities on agricultural properties. My focus was on ways to represent large and complex GIS information like polygons of plantation areas, satellite image, pests and disease occurrence of farmlands in a mobile maps application. Besides developed functionality mobile map piece, tested and debugged REST API services and used design patterns and refactoring to increase bugs solution and code readability, I travelled to some farms to get up close the complete user experience.

\sectionsep

\runsubsection{\href{http://imazon.org.br/}{Imazon}}
\descript{| Android Mobile Development Trainee}
\location{July - Dez 2013 | Belém, PA}
\sectionsep 

I worked with IMAZON’s android app development in the Eclipse and Android Studio development environments. My focus was on developing the app using Android Google Maps API, GreenDAO, and RETROFIT framework.

\sectionsep

\runsubsection{\href{http://imazon.org.br/}{Imazon}}
\descript{| Trainee Web Developer}
\location{June 2012 - July 2013 | Belém, PA}
\sectionsep 

I worked as Full-Stack programmer in the Python web framework Django, and was responsible for setting up the web services for the green municipalities project. I was programming the database manager, the domain logic of application to improve user’s dataset view like annual deforestation, water access and per capita income of Pará state. And the build of a web page for municipality statistics with help of tools like google charts, Jquery and  AJAX.

\sectionsep

\end{minipage} 

\end{document} % End of resume.
\documentclass{article}

\textit{•}